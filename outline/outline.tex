\documentclass{article}
\usepackage{amssymb}
\usepackage{amsmath}
\usepackage{outlines}
\usepackage{enumitem}

\begin{document}
\title{An Overview of Riemann-Hurwitz in Positive Characteristic}
\author{Dean Bisogno}
\maketitle
\setenumerate[1]{label=\Roman*.}
\setenumerate[2]{label=\Alph*.}
\setenumerate[3]{label=\roman*.}
\setenumerate[4]{label=\alph*.}
\begin{outline}[enumerate]
   \1 A Reminder of Riemann-Hurwitz from characteristic 0
   		\2 $2g_{X} - 2 = d(2g_{Y} -2) + \sum_{x \in \mathcal{X}} \nu_b$ 
   			\3 Where $g_x$ and $g_y$ are the genuses of the the respective spaces, $d$ the degree of the covering map $X \to Y$, $\nu_x = e_x - 1$ the differential length of $x$ and $e_x$ the ramification index of $x$.
   \1 Now think about algebraically closed field of characteristic p. An example that still works.
   		\2 An Artin-Schreier Curve: $X^{\circ}_{p,t}: x^p - x -y^t$ where $p \not | t$.
   		\2 For $\pi_{x,p,t}: X^{\circ}_{p,t} \to \mathbb{A}^{1}_{k}$ projection on $x$, ramified only over the $p$ points $(x,0)$, each with $\nu_{(x,0)} = t-1$.
   		\2 Can calculate a genus for $X$ with Riemann-Hurwitz, and we get $\frac{(p-1)(t-1)}{2}$. Can verify that by checking geometric genus $dim(H^{0}(X,\Omega^{1})$. Can find basis $\{y^b x^r dy | r \geq 0, b \geq 0, bp + rt \geq (p - 1)(t - 1) -2\}$
   \1 A Case that breaks
         \2 For $\pi_{y,p,t}: X_{p,t} \to \mathbb{P}^{1}_{k}$ (projection on y) only ramified over $P_{\infty}$ with ramification index $p$.
         \2 Classical Riemann-Hurwitz: $2g_X - 2 = p(-2)+(p-1)$ yields a negative genus!
         \2 Need to be careful in the case $p | e_x$
   \1 Wild Riemann-Hurwitz
   	     \2 For a non-constant, separable, Galois cover $f:X \to Y$ of smooth projective curves over field $k$ of characteristic $p$. Let $e_x$ be the ramification index of $f$ at a point $x$. And if $p \not | e_x$ then $\nu_x = e_x -1$. If $p | e_x$ then there exists a ramification filtration of the inertia group $I$ at $x$ such that $\nu_x = \sum_{i=0}^{\infty}(|I_i| - 1)$. Then
   	     $$
   		 2g_{X} - 2 = d(2g_{Y} -2) + \sum_{x \in \mathcal{X}} \nu_b 
   	     $$
   	     \2 If $e_x = p$ then $\nu_x = (p-1)(t_x + 1)$ where $t_x$ is the ramification jump. Understanding the ramification jump requires higher ramification groups. The ramification jumps occur where the filtration of the ramification groups in upper numbering yields strict inclusion.
           \2 Higher ramification groups.
               \3 For Galois extension $L$ or $K$ with Galois groups $G$, where $K$ is complete with respect to discrete valuation $\nu_K$ and $L$ has the discrete valuation induced by $\nu_K$ (there are some normalization conditions on the discrete valuations). Then we can define the $i^{th}$ ramification group by
               $$
               I_i = \{\sigma \text{ in } G \text{ | }\nu_L(\sigma(a) - a) \geq i + 1 \; \forall a \in \mathcal{O}_L \}.
               $$
            \2 We see that there is a natural filtration of the higher ramification groups. The with the inertia group corresponding to $I_0$. There is another chain  of inclusions for the higher ramification groups in the upper numbering, which is likely beyond the scope of the presentation.
   \1 Finish ``a case that breaks,'' show that $t_{P_{\infty}} = 1$, then Wild-Riemann Hurwitz calculates the correct genus
   \1 If time is permitting or if we condense the beginning, perhaps we can looks at a case where $p| e_x$ and $p \not = e_x$.
\end{outline}
\end{document}