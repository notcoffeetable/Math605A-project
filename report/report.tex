%%%%%%%%%%%%%%%%%%%%%%%%%%%%%%%%%%%%%%%%%
% Short Sectioned Assignment
% LaTeX Template
% Version 1.0 (5/5/12)
%
% This template has been downloaded from:
% http://www.LaTeXTemplates.com
%
% Original author:
% Frits Wenneker (http://www.howtotex.com)
%
% License:
% CC BY-NC-SA 3.0 (http://creativecommons.org/licenses/by-nc-sa/3.0/)
%
%%%%%%%%%%%%%%%%%%%%%%%%%%%%%%%%%%%%%%%%%

%----------------------------------------------------------------------------------------
%	PACKAGES AND OTHER DOCUMENT CONFIGURATIONS
%----------------------------------------------------------------------------------------

\documentclass[paper=a4, fontsize=11pt]{scrartcl} % A4 paper and 11pt font size

\usepackage[T1]{fontenc} % Use 8-bit encoding that has 256 glyphs
\usepackage{fourier} % Use the Adobe Utopia font for the document - comment this line to return to the LaTeX default
\usepackage{amsmath,amsfonts,amsthm} % Math packages
\usepackage[american]{babel}
\usepackage{csquotes}
\usepackage[style=apa,sortcites=true,sorting=nyt,backend=biber]{biblatex} 
\DeclareLanguageMapping{american}{american-apa}%bibliography package

\usepackage{sectsty} % Allows customizing section commands
\allsectionsfont{\centering \normalfont\scshape} % Make all sections centered, the default font and small caps

\usepackage{fancyhdr} % Custom headers and footers
\pagestyle{fancyplain} % Makes all pages in the document conform to the custom headers and footers
\fancyhead{} % No page header - if you want one, create it in the same way as the footers below
\fancyfoot[L]{} % Empty left footer
\fancyfoot[C]{} % Empty center footer
\fancyfoot[R]{\thepage} % Page numbering for right footer
\renewcommand{\headrulewidth}{0pt} % Remove header underlines
\renewcommand{\footrulewidth}{0pt} % Remove footer underlines
\setlength{\headheight}{13.6pt} % Customize the height of the header

\usepackage{enumitem}
\numberwithin{equation}{section} % Number equations within sections (i.e. 1.1, 1.2, 2.1, 2.2 instead of 1, 2, 3, 4)
\numberwithin{figure}{section} % Number figures within sections (i.e. 1.1, 1.2, 2.1, 2.2 instead of 1, 2, 3, 4)
\numberwithin{table}{section} % Number tables within sections (i.e. 1.1, 1.2, 2.1, 2.2 instead of 1, 2, 3, 4)

\setlength\parindent{0pt} % Removes all indentation from paragraphs - comment this line for an assignment with lots of text
\bibliography{biblio}
%----------------------------------------------------------------------------------------
%	TITLE SECTION
%----------------------------------------------------------------------------------------

\newcommand{\horrule}[1]{\rule{\linewidth}{#1}} % Create horizontal rule command with 1 argument of height
\newtheoremstyle{break}
  {\topsep}{\topsep}%
  {\itshape}{}%
  {\bfseries}{.}%
  {\newline}{}%
\theoremstyle{break}
\newtheorem{lem}{Lemma}
\newtheorem{defn}{Definition}
\newtheorem{thm}{Theorem}
\newtheorem{cor}{Corollary}
\newtheorem{prop}{Proposition}
\newtheorem{ex}{Example}
\renewcommand\qedsymbol{//}
\newtheorem{lma}{Lemma}
\setenumerate[1]{label=\Roman*.}
\setenumerate[2]{label=\Alph*.}
\setenumerate[3]{label=\roman*.}
\setenumerate[4]{label=\alph*.}

\title{	
\normalfont \normalsize 
\textsc{Colorado State University Mathematics} \\ [25pt] % Your university, school and/or department name(s)
%\horrule{0.5pt} \\[0.4cm] % Thin top horizontal rule
\huge Higher Ramification Groups \\ % The assignment title
%\horrule{2pt} \\[0.5cm] % Thick bottom horizontal rule
}

\author{Dean Bisogno} % Your name

\date{\normalsize\today} % Today's date or a custom date

\begin{document}

\maketitle % Print the title
\section{Abstract}
Studying higher ramification groups immediately depends on some key ideas from valuation theory. With that in mind we hope to layout the essential results from valuation theory before proceeding to the subject of this paper. Higher ramification groups arise when studying extensions of fields, and the ramification of primes in subrings of the base field. In particular ramification groups can be useful when studying wildly ramified covers (when the characteristic of your base field divides the ramification index of some prime). Then need to find the order of higher ramification groups to correct our ramification index. Another application of higher ramification groups is to study the subgroup structure of a Galois group. As the higher ramifications groups will all be subgroups, and in some instances tell us about sylow p-subgroup structure. The aim of this paper is to understand at a general level what higher ramification groups are, and investigate a couple examples of their applications.
\section{Preliminaries}
\subsection{Valuations}
We will primarily deal with discrete valuations, though
\begin{defn}[Discrete Valuation Ring (D.V.R.)]
A discrete valuation ring is a principal ideal domain $O$ with a unique maximal ideal $p \not= 0$.
\end{defn}
\begin{defn}[Uniformizing Parameter]
Let $\mathfrak{p}$ be the unique maximal ideal of D.V.R. $O$. Since $O$ is a PID, there exists a prime $\pi$ in $O$ such that $\mathfrak{p}=(\pi)$. We call $\pi$ a uniformizing parameter.
\end{defn}
Theorem 7, section 16.2 in (\cite{DnF}) shows that $\pi$ is unique (up to multiplication by a unit) and generates the unique maximal ideal of $O$. Further in problem 26, in section 7.1 of \cite{DnF} (which we proved in math 566) we showed that $O \setminus \mathfrak{p}$ is the set of units in $O$. Then for any $a\in O\setminus \{0\}$ we know that $(a)$ is an ideal of $O$. We know $(a)$ is a nonzero ideal of $O$ and because $(\pi)$ is maximal, and in local ring $O$, there exists some $n\in \mathbb{Z}$ such that $(a) = (\pi)^n$. Notice if $a$ is a unit, then $n=0$ because $<a>=O$. 
\begin{defn}[Discrete Valuation]
The exponent $n$ used above is the valuation of $a$ denoted $v_\mathfrak{p}(a)$, it is a function $v_\mathfrak{p}:K^* \to \mathbb{Z}$ where $K$ is the field of fractions of $O$. We can extend $v_\mathfrak{p}$ to $K$ by setting $v_\mathfrak{p}(0) = \infty$. Further $v_\mathfrak{p}$ satisfies the following
\begin{enumerate}
\item $v_\mathfrak{p}(ab)=v_\mathfrak{p}(a) + v_\mathfrak{p}(b)$.
\item $v_\mathfrak{p}(a+b) \geq min\{v_\mathfrak{p}(a),v_\mathfrak{p}(b) \}$.
\end{enumerate}
\end{defn}
There are also valuations which are not discrete, and we can classify all valuations as either archimedean or nonarchimedean. We call a valuation $v$ nonarchimedean if $v(n)$ is bounded for every $n\in \mathbb{N}$. Proposition 3.6 in \cite{Neukirch} is often incorporated in the definition of discrete valuations, as every discrete valuation is nonarchimedean.
\begin{prop}
 A valuaion $v(x)$ is nonarchimedean if and only if it satisfies $v(x+y) \leq max\{v(x),v(y)\}$.
\end{prop}
\begin{proof}
$(\Leftarrow)$ The reverse direction is straight forward. We just notice 
$$v(n)=v(1+\ldots+1) \leq 1\; \forall n\in\mathbb{N}.$$

$(\Rightarrow)$ Now, $v(n) \leq N$ for all $n\in\mathbb{N}$ for some $N \in \mathbb{N}$. Then for arbitrary $x,y\in K$, without loss of generality, consider $v(x) \geq v(y)$. Choose $l \geq 0$, so we get $v(x)^{l}v(y)^{n-l} \leq v(x)$. Now applying binomial formula we see that
$$
	v(x+y)^{n} \leq \sum_{l=0}^{n} v\tbinom{n}{l} v(x)^l v(y)^{n-l} \leq N^{}(n+1)(v(x))^n
$$
taking the $n^{th}$ root of both sides yields
$$
v(x+y) \leq N^{\frac{1}{n}}(1+n)^{\frac{1}{n}} v(x) = N^{\frac{1}{n}}(1+n)^{\frac{1}{n}} max\{v(x), v(y) \}
$$

The result then follows if we let $n \to \infty$. We can conclude then that for any discrete valuation (which is nonarchimedean) $v(x+y) \leq max \{v(x),v(y) \}$.
\end{proof}

Valuations are particularly useful when studying number fields because of the following propositions
\begin{prop}[\cite{Neukirch}]
Let $O$ be a noetherian integral domain. $O$ is a Dedekind domain if and only if, for all prime ideals $\mathfrak{p} \not= 0$, the localizations $O_\mathfrak{p}$ are discrete valuation rings.
\end{prop}
\begin{prop}[\cite{Neukirch}]
Every valuation of $\mathbb{Q}$ is equivalent to one of $\nu_p(x)$ (the nonarchimedean valuations) or $\nu_\infty(x)$ (the archimedean valuations).
\end{prop}

\begin{defn}[Prolongation of a valuation]
If $A \subset B$ are rings with $B$ integral over $A$, $\mathfrak{p}$ a prime in $A$, and $\mathfrak{q}$ prime in $B$ dividing $\mathfrak{p}$ with ramification index $e_\mathfrak{q}$ then $v_\mathfrak{q}(x)=e_\mathfrak{q}v_\mathfrak{p}(x)$ is a prolongation of $v_\mathfrak{p}$ with index $e_\mathfrak{q}$ (\cite{Serre}).
\end{defn}

\subsection{Field Completion}
With a valuation $\nu$ on a field $K$ then for any real number $a\in(0,1)$ we can induce an absolute value on $K$ by

\[ ||x|| =  \begin{cases} 
      a^{\nu(x)} & x \not= 0 \\
      0 & x = 0 \\
   \end{cases}
\]
which satisfies the usual conditions of a metric as outlined by. A topology is then induced on $K$ via the absolute value metric, and we can denote the completion of $K$ with respect to the valuation $\nu$ by $K_\nu$. Note also that the metrics induced by different choices of $a$ are topologically equivalent, so the completion is dependent only on $\nu$. Further, $\nu$ extends in the completion of $K$ to a discrete valuation (which we will continue to call $\nu$) on $K_\nu$ (\cite{Serre}).

\section{Higher Ramification Groups}
Consider now a finite Galois extension $L|K$ with Galois group $\Gamma$, associated discrete valuations $\omega$ on $L$ and $\nu$ on $K$ with uniformizing parameters $\pi_L$ and $\pi_K$ respectively, and integer rings $O_L$ and $O_K$.

Recall that 
\begin{defn}[Higher Ramification Groups, \cite{Serre}]
For every real number $i > -1$ we define the $i^{th}$ ramification group of $L|K$ by
$$
G_i = G_i(L|K) = \{\sigma \in \Gamma | \omega(\sigma(a)-a)\geq i+1 \;\forall a \in O_L\}.
$$
\end{defn}
The following is a proposition which helps characterize the higher ramification groups
\begin{prop}
For any $a \in O_L$ and $\sigma \in G$, $\omega(\sigma(a) - a) \geq i+1$ if and only if  $\sigma(a) \equiv a \; mod \; <\pi_L>^{i+1}$.
\end{prop}
\begin{proof}

($\Rightarrow$)
$$
\nu_L(\sigma(a)-a) \geq i+1 \implies \sigma(a)-a = \pi_L^(t)\frac{a}{b}
$$
where $t \geq i+1$ and $a,b,\pi_L\in O_L$ all relatively prime. The above then implies $\sigma(a)-a \equiv 0 \; mod \; <\pi_L>^{i+1}$.

($\Leftarrow$) $$\sigma(a)-a \equiv 0 \; mod \; \pi_L^{i+1} \implies \sigma(a)-a = \pi_L^{t}\frac{a}{b}$$
with the same restrictions as above. It follows then that $\nu_L(\sigma(a)-a) \geq i+1$.
\end{proof}
Clearly this induces the following filtration on G:
$$
\{G_i(L|K)\}_{i\geq -1} = G_{-1} \supseteq G_0 \supseteq G_1 \supseteq \ldots
$$


\begin{prop}
The first two ramification groups are $G_{0} = G$ and $G_{1}=I$.
\end{prop}

\begin{prop}
For $L|K$ is Galois with group $\Gamma$, $\nu_\mathfrak{p}(x)$ a valuation on $K$ and $\omega_\mathfrak{q}(x)$ a prolongation of $\nu_\mathfrak{p}$, let $\widehat{L}$,$\widehat{K}$ be the completions of $L$ and $K$ with respect to $\omega_\mathfrak{q}$ and $\nu_\mathfrak{p}$ respectively. If $D(L|K)$ is the decomposition group of $\mathfrak{q}$ over $\mathfrak{p}$, then $\widehat{L}|\widehat{K}$ is Galois with Galois group $D(L|K)$.
\end{prop}

\section{The Upper Numbering and Ramification Jump}

\begin{defn}[Upper Numbering of the Higher Ramification groups]
Consider the function
$$
t = \varphi(s) = \int_{0}^{s} \frac{dx}{[G_0 : G_x]}
$$
called the Herbrand function which has inverse map $\psi$.
Then for any real $s \geq -1$ let $G_s = G_{\lceil s \rceil}$ and renumber the ramification groups by $G^t(L|K) = G_s(L|K)$ where $s=\psi(t)$. 
\end{defn}
\begin{defn}[Ramification Jump]
If $G^{t}(L|K) \not=G^{t+\epsilon}(L|K)$ for any $\epsilon \geq 0$ then we call $t$ a ramification jump.
\end{defn}

\begin{thm}[Hasse-Arf]
For a finite abelian extension $L|K$, the jumps of the filtration $\{G^{i}(L|K)\}_{i \geq -1}$ are rational integers.
\end{thm}
Serre gives a useful interpretation of the Hasse-Arf theorem, namely that if $G_i \not= G_{i+1}$ then $\varphi(i)$ is an integer. We can use this in the following example (\cite{Serre}).

\begin{ex}
Suppose $G$ is a cyclic group of order $p^n$ where $p$ is the characteristic of $\widehat{K}$. Let $G(i)$ be the subgroup of $G$ with order $p^{n-i}$. Then there exist integers $i_0, \ldots,i_{n-1}$ such that we can identify all ramification groups as follows:
$$G_0 = G_{i_0} = G = G^0 = G^{i_0}$$
$$G_{{i_0}+1} = \ldots = G_{{i_0}+pi_1} = G(1) = G^{i_0+1} = \ldots = G^{i_0 + i_1}$$
$$G_{{i_0}+pi_1+1} = \ldots = G_{{i_0}+pi_1+p^2 i_2} = G(2) = G^{i_0+i_1+1} = \ldots = G^{i_0 + i_1+i_2}$$
$$\vdots$$
$$G_{i_0+pi_1+p^2 i_2 + \cdots + p^{n-1}i_{n-1} +1}={1}=G^{i_0+\cdots+i_{n-1}+1}.$$
\end{ex}

\section{Applications}
\subsection{Cyclotomic Extensions of $\mathbb{Q}_p$}
In this example we consider the p-adic completion of $\mathbb{Q}$ with respect to a valuation (and associated metric) $\nu_p$. Let $n=p^m$, and $\zeta$ be a primitive $n^{th}$ root of unity and the extension $\mathbb{Q}_p(\zeta)|\mathbb{Q}_p$ with Galois group $G$. We recall that $[\mathbb{Q}_p(\zeta):\mathbb{Q}_p]=\phi(n)=(p-1)p^{m-1}$ and $G\cong (\mathbb{Z}/n)^*$. 

Now, if $0 \leq v \leq m$ then let $G^v$ be the subgroup of $G$ isomorphic to the subgroup $H \subset (\mathbb{Z}/n)^*$ such that $a \equiv 1\;mod\;p^v$ for every $a\in H$. Since $Gal(\mathbb{Q}_p(\zeta_{p^v})|\mathbb{Q}_p) \cong (\mathbb{Z}/p^v)^*$ we can see that $Gal(\mathbb{Q}_p(\zeta_{p^m})|\mathbb{Q}_p(\zeta_{p^v})) \cong G^v$. Using this fact we can find all ramification groups $G_i$ of $G$ (\cite{Serre}).
\begin{prop}
The ramification groups $G_i$ of $G$ are
\[ G_u = \begin{cases} 
      G    & u=0 \\
      G^1  & 1 \leq u \leq p -1 \\
      G^2  & p \leq u \leq p^2 -1\\
      \vdots \\
      \{1\} & p^{m-1} \leq u
   \end{cases}
\]
\end{prop}

\begin{proof}
\end{proof}

\subsection{Artin-Schreier Extensions in Positive Characteristic}
(example from char p appendix to Renzo's book)
We can rephrase Riemann-Hurwitz such that if field $\mathbb{F}$ has characteristic $p$ and $p|e$ where $e$ is the ramification index of a point under a degree $d$ covering map $\phi:X \to Y$ of curves over $\mathbb{F}$, then
$$
2g_Y -2 = d(2g_X -2) + \sum_{x\in X} \big( \sum^{\infty}_{i=0}|G_i(x)| -1 \big).
$$

For a smooth compact algebraic curve $X_{p,t}: x^p - x - f(y)$ where $f(y)$ is degree $t$ over algebraically closed field $\mathbb{F}$ of characteristic $p \not= 0$. Now consider the covering map $\psi: X_{p,t} \to \mathbb{A}^1_\mathbb{F}$ of the affine line over $\mathbb{F}$ such that $\psi(x,y)=y$. 

Strategy (use change of variables $x=x\bar{y}^a$,$\bar{y}=1/y$ so $p_\infty = (0,0)$, and $t=ap-1$):
Since $\psi$ is clearly unramified over $\mathbb{A}^1_\mathbb{F}$, consider now $\psi: X_{p,t} \to \mathbb{P}^2_\mathbb{F}$. We see now that $\psi$ is ramified over $p_\infty$ and $Gal(\psi)\cong \mathbb{Z}/p = <\tau>=<x+1>$. We want to show that there exists uniformizing parameter $\pi$ such that $v_\infty(\pi) = 1$, then since $\tau=x+1$ generates $G$ we want to check $\tau(\pi)-\pi$, and finally evaluate $t_\infty = v_{\infty}(\tau(\pi)-\pi)$. Because $\mathbb{Z}/p = G$ is simple $G_{t_\infty}={1}$ so we get 
$$
\sum_{x\in X} \big( \sum^{\infty}_{i=0}|G_i(x)| -1 \big) = |\mathbb{Z}/p|(t_{p_\infty} + 1)=(p-1)(t_{p_\infty} + 1).
$$
Which we can confirm by \cite{stitchenoth} III.7.8.
\nocite{*}
\printbibliography


\end{document}