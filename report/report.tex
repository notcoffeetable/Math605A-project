%%%%%%%%%%%%%%%%%%%%%%%%%%%%%%%%%%%%%%%%%
% Short Sectioned Assignment
% LaTeX Template
% Version 1.0 (5/5/12)
%
% This template has been downloaded from:
% http://www.LaTeXTemplates.com
%
% Original author:
% Frits Wenneker (http://www.howtotex.com)
%
% License:
% CC BY-NC-SA 3.0 (http://creativecommons.org/licenses/by-nc-sa/3.0/)
%
%%%%%%%%%%%%%%%%%%%%%%%%%%%%%%%%%%%%%%%%%

%----------------------------------------------------------------------------------------
%	PACKAGES AND OTHER DOCUMENT CONFIGURATIONS
%----------------------------------------------------------------------------------------

\documentclass[paper=a4, fontsize=11pt]{scrartcl} % A4 paper and 11pt font size

\usepackage[T1]{fontenc} % Use 8-bit encoding that has 256 glyphs
\usepackage{fourier} % Use the Adobe Utopia font for the document - comment this line to return to the LaTeX default
\usepackage{amsmath,amsfonts,amsthm} % Math packages
\usepackage[american]{babel}
\usepackage{csquotes}
\usepackage[style=apa,sortcites=true,sorting=nyt,backend=biber]{biblatex} 
\DeclareLanguageMapping{american}{american-apa}%bibliography package

\usepackage{sectsty} % Allows customizing section commands
\allsectionsfont{\centering \normalfont\scshape} % Make all sections centered, the default font and small caps

\usepackage{fancyhdr} % Custom headers and footers
\pagestyle{fancyplain} % Makes all pages in the document conform to the custom headers and footers
\fancyhead{} % No page header - if you want one, create it in the same way as the footers below
\fancyfoot[L]{} % Empty left footer
\fancyfoot[C]{} % Empty center footer
\fancyfoot[R]{\thepage} % Page numbering for right footer
\renewcommand{\headrulewidth}{0pt} % Remove header underlines
\renewcommand{\footrulewidth}{0pt} % Remove footer underlines
\setlength{\headheight}{13.6pt} % Customize the height of the header

\usepackage{enumitem}
\numberwithin{equation}{section} % Number equations within sections (i.e. 1.1, 1.2, 2.1, 2.2 instead of 1, 2, 3, 4)
\numberwithin{figure}{section} % Number figures within sections (i.e. 1.1, 1.2, 2.1, 2.2 instead of 1, 2, 3, 4)
\numberwithin{table}{section} % Number tables within sections (i.e. 1.1, 1.2, 2.1, 2.2 instead of 1, 2, 3, 4)

\setlength\parindent{0pt} % Removes all indentation from paragraphs - comment this line for an assignment with lots of text
\bibliography{biblio}
%----------------------------------------------------------------------------------------
%	TITLE SECTION
%----------------------------------------------------------------------------------------

\newcommand{\horrule}[1]{\rule{\linewidth}{#1}} % Create horizontal rule command with 1 argument of height
\newtheoremstyle{break}
  {\topsep}{\topsep}%
  {\itshape}{}%
  {\bfseries}{.}%
  {\newline}{}%
\theoremstyle{break}
\newtheorem{lem}{Lemma}
\newtheorem{defn}{Definition}
\newtheorem{thm}{Theorem}
\newtheorem{cor}{Corollary}
\newtheorem{prop}{Proposition}
\newtheorem{ex}{Example}
\renewcommand\qedsymbol{//}
\newtheorem{lma}{Lemma}
\setenumerate[1]{label=\Roman*.}
\setenumerate[2]{label=\Alph*.}
\setenumerate[3]{label=\roman*.}
\setenumerate[4]{label=\alph*.}

\title{	
\normalfont \normalsize 
\textsc{Colorado State University Mathematics} \\ [25pt] % Your university, school and/or department name(s)
%\horrule{0.5pt} \\[0.4cm] % Thin top horizontal rule
\huge Higher Ramification Groups \\ % The assignment title
%\horrule{2pt} \\[0.5cm] % Thick bottom horizontal rule
}

\author{Dean Bisogno} % Your name

\date{\normalsize\today} % Today's date or a custom date

\begin{document}

\maketitle % Print the title
\section{Abstract}
Studying higher ramification groups immediately depends on some key ideas from valuation theory. With that in mind we hope to layout the essential results from valuation theory before proceeding to the subject of this paper. Higher ramification groups arise when studying extensions of fields, and the ramification of primes in the base field. This in particular appears when studying the genus or fundamental group of spaces. From Riemann-Hurwitz we can calculate the genus of a space if we know the degree of the appropriate cover or extension, and ramification of such a cover. In well behaved situations, the ramification information can be known independently or just read off from the ramification index. But in more interesting situations (such as when the characteristic of your base field divides the ramification index), we need to inspect the order of higher ramification group to correct our ramification index. Another application of higher ramification groups is to study the subgroup structure of a Galois group, as the higher inertia groups will all be subgroups, and particular inertia groups will yield information about sylow p-subgroups of the Galois group in question. We hope to understand at a broad level what higher ramification groups are, and investigate a couple examples of their use.
\section{Preliminaries}
\subsection{Valuations}
We borrow our definitions in this section from \cite{Neukirch}. 
\begin{defn}[Valuations]
A valuation on a field $K$ is a function $\nu_K: K \to \mathbb{R}$ such that
\begin{enumerate}
\item $\nu_K(x) \geq 0$, and $\nu_K(x)= 0 \Leftrightarrow x =0$
\item $\nu_K(xy) = \nu(x)\nu(y)$.
\item $\nu_K(x+y) \leq \nu_K(x) + \nu_K(y)$.
\end{enumerate}
\end{defn}
\begin{defn}[Nonarchimedean]
If $\nu_K(n)$ is bounded for all $n \in \mathbb{N}$ then we call $\nu_K$ nonarchimedean. Otherwise it is archimedean.
\end{defn}

Proposition 3.6 in \cite{Neukirch} shows that $\nu_K$ is nonarchimedean if and only if it satisfies $\nu_{K}(x+y) \leq max\{\nu_K(x),\nu_K(y)\}$. This is commonly incorporated in the definition of a discrete valuation as in \cite{DnF}. 
\begin{proof}

$(\Leftarrow)$ The reverse direction is straight forward. We just notice 
$$\nu_K(n)=\nu_K(1+\ldots+1) \leq 1\; \forall n\in\mathbb{N}.$$

$(\Rightarrow)$ Now, $\nu_K(n) \leq N$ for all $n\in\mathbb{N}$ for some $N \in \mathbb{N}$. Then for arbitrary $x,y\in K$, without loss of generality, consider $\nu_K(x) \geq \nu_K(y)$. Choose $l \geq 0$, so we get $\nu_K(x)^{l}\nu_K(y)^{n-l} \leq \nu_K(x)$. Now applying binomial formula we see that
$$
	\nu_K(x+y)^{n} \leq \sum_{l=0}^{n} \nu_K({{n}\choose{l}}) \nu_K(x)^l \nu_K(y)^{n-l} \leq N^{}(n+1)(\nu_K(x))^n
$$
taking the $n^{th}$ root of both sides yields
$$
\nu_K(x+y) \leq N^{\frac{1}{n}}(1+n)^{\frac{1}{n}} \nu_K(x) = N^{\frac{1}{n}}(1+n)^{\frac{1}{n}} max\{\nu_K(x), \nu_K(y) \}
$$

The result then follows if we let $n \to \infty$.
\end{proof}

Valuations are particularly useful when studying number fields because of the following proposition
\begin{prop}[\cite{Serre}]
Every valuation of $\mathbb{Q}$ is equivalent to one of $|\cdot|_p$ (the nonarchimedean valuations) or $|\cdot|_\infty$ (the archmidean valuations).
\end{prop}


\begin{defn}[Discrete Valuation Ring (D.V.R.)]
A discrete valuation ring is a principal ideal domain $O$ with a unique maximal ideal $p \not= 0$.
\end{defn}
\begin{defn}[Uniformizing Parameter]
If $O$ is a D.V.R. of $K$ via $\nu_K$, then an element $\pi \in R$ such that $\nu_K(\pi) = 1$ is a uniformizing parameter for $O$. Theorem 7, section 16.2 in (\cite{DnF}) shows that $\pi$ is unique (up to multiplication by a unit) and generates the unique maximal ideal of $O$.
\end{defn}

For a finite Galois extension $L|K$ with with valuation rings $O_L$ and $O_L$, and prime $P_K\in A_K$, then $P_K A_L$ is a proper ideal in $A_L$. Since $A_L$ is local, then $P_K A_L \subset P_L$ the unique maximal ideal of $A_L$. Because all ideal in $A_L$ are some power of $P_L$, this means that $P_K A_L = P_L^e$ for some integer $e$.

\subsection{Field Completion}
With a valuation $\nu_K$ on a field $K$ then for any real number $a\in(0,1)$ we can induce an absolute value on $K$ by

\[ ||x|| =  \begin{cases} 
      a^{\nu_K(x)} & x \not= 0 \\
      0 & x = 0 \\
   \end{cases}
\]
which satisfies the usual conditions of a metric as outlined by (\cite{Serre}). A topology is then induced on $K$ via the absolute value metric, and we can denote the completion of $K$ with respect to the valuation $\nu_K$ by $K_\nu$. Note also that the metrics induced by different choices of $a$ are topologically equivalent, so the completion is dependent only on $\nu_K$. Further, $\nu_K$ extends in the completion of $K$ to a discrete valuation (which we will continue to call $\nu_K$) on $K_\nu$ (\cite{Serre}).

\section{Higher Ramification Groups}
Consider now a finite Galois extension $L|K$ with Galois group $G$, and associated discrete valuations $\nu_K$ and $\omega_L$ with uniformizing parameters $\pi_L$ and $\pi_K$.

\begin{defn}[Higher Ramification Groups]
For every real number $i > -1$ we define the $i^{th}$ ramification group of $L|K$ by
$$
G_i = G_i(L|K) = \{\sigma \in G | \nu_L(\sigma(a)-a)\geq i+1 \;\forall a \in O_L\}.
$$
\end{defn}

\begin{prop}
For any $a \in O_L$ and $\sigma \in G$, $\nu_L(\sigma(a) - a) \geq i+1$ if and only if  $\sigma(a) \cong a \; mod \; \pi_L^{i+1}$.
\end{prop}
\begin{proof}

($\Rightarrow$)
$$
\nu_L(\sigma(a)-a) \geq i+1 \implies \sigma(a)-a = \pi_L^(t)\frac{a}{b}
$$
where $t \geq i+1$ and $a,b,\pi_L\in O_L$ all relatively prime. The above then implies $\sigma(a)-a \cong 0 \; mod \; \pi_L^{i+1}$.

($\Leftarrow$) $$\sigma(a)-a \cong 0 \; mod \; \pi_L^{i+1} \implies \sigma(a)-a = \pi_L^{t}\frac{a}{b}$$
with the same restrictions as above. It follows then that $\nu_L(\sigma(a)-a) \geq i+1$.
\end{proof}
Clearly this induces the following filtration on G:
$$
\{G_i(L|K)\}_{i\geq -1} = G_{-1} \supseteq G_0 \supseteq G_1 \supseteq \ldots
$$



\begin{prop}
The first two ramification groups are $G_{0} = G$ and $G_{1}=I$.
\end{prop}

\section{The Upper Numbering and Ramification Jump}

\begin{defn}[Upper Numbering of the Higher Ramification groups]
Consider the function
$$
t = \varphi(s) = \int_{0}^{s} \frac{dx}{[G_0 : G_x]}
$$
called the Herbrand function which has inverse map $\psi$.
Then for any real $s \geq -1$ let $G_s = G_{\lceil s \rceil}$ and renumber the ramification groups by $G^t(L|K) = G_s(L|K)$ where $s=\psi(t)$. 
\end{defn}
\begin{defn}[Ramification Jump]
If $G^{t}(L|K) \not=G^{t+\epsilon}(L|K)$ for any $\epsilon \geq 0$ then we call $t$ a ramification jump.
\end{defn}
\begin{thm}[Hasse-Arf]
For a finite abelian extension $L|K$, the jumps of the filtration $\{G^{i}(L|K)\}_{i \geq -1}$ are rational integers.
\end{thm}
\begin{ex}
\end{ex}

\subsection{Application to Wild Riemann-Hurwitz}
We can rephrase Riemann-Hurwitz such that if field $\mathbb{F}$ has characteristic $p$ and $p|e$ where $e$ is the ramification index of a point under a degree $d$ covering map $\phi:X \to Y$ of curves over $F$
Then
$$
2g_Y -2 = d(2g_X -2) + \sum_{x\in X} \big( \sum^{\infty}_{i=0}|G_i(x)| -1 \big)
$$
\nocite{*}
\printbibliography


\end{document}