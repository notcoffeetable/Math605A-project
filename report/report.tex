%%%%%%%%%%%%%%%%%%%%%%%%%%%%%%%%%%%%%%%%%
% Short Sectioned Assignment
% LaTeX Template
% Version 1.0 (5/5/12)
%
% This template has been downloaded from:
% http://www.LaTeXTemplates.com
%
% Original author:
% Frits Wenneker (http://www.howtotex.com)
%
% License:
% CC BY-NC-SA 3.0 (http://creativecommons.org/licenses/by-nc-sa/3.0/)
%
%%%%%%%%%%%%%%%%%%%%%%%%%%%%%%%%%%%%%%%%%

%----------------------------------------------------------------------------------------
%	PACKAGES AND OTHER DOCUMENT CONFIGURATIONS
%----------------------------------------------------------------------------------------

\documentclass[paper=a4, fontsize=11pt]{scrartcl} % A4 paper and 11pt font size

\usepackage[T1]{fontenc} % Use 8-bit encoding that has 256 glyphs
\usepackage{fourier} % Use the Adobe Utopia font for the document - comment this line to return to the LaTeX default
\usepackage{amsmath,amsfonts,amsthm} % Math packages
\usepackage[american]{babel}
\usepackage{csquotes}
\usepackage[style=apa,sortcites=true,sorting=nyt,backend=biber]{biblatex} 
\DeclareLanguageMapping{american}{american-apa}%bibliography package

\usepackage{sectsty} % Allows customizing section commands
\allsectionsfont{\centering \normalfont\scshape} % Make all sections centered, the default font and small caps

\usepackage{fancyhdr} % Custom headers and footers
\pagestyle{fancyplain} % Makes all pages in the document conform to the custom headers and footers
\fancyhead{} % No page header - if you want one, create it in the same way as the footers below
\fancyfoot[L]{} % Empty left footer
\fancyfoot[C]{} % Empty center footer
\fancyfoot[R]{\thepage} % Page numbering for right footer
\renewcommand{\headrulewidth}{0pt} % Remove header underlines
\renewcommand{\footrulewidth}{0pt} % Remove footer underlines
\setlength{\headheight}{13.6pt} % Customize the height of the header

\usepackage{enumitem}
\numberwithin{equation}{section} % Number equations within sections (i.e. 1.1, 1.2, 2.1, 2.2 instead of 1, 2, 3, 4)
\numberwithin{figure}{section} % Number figures within sections (i.e. 1.1, 1.2, 2.1, 2.2 instead of 1, 2, 3, 4)
\numberwithin{table}{section} % Number tables within sections (i.e. 1.1, 1.2, 2.1, 2.2 instead of 1, 2, 3, 4)

\setlength\parindent{0pt} % Removes all indentation from paragraphs - comment this line for an assignment with lots of text
\bibliography{biblio}
%----------------------------------------------------------------------------------------
%	TITLE SECTION
%----------------------------------------------------------------------------------------

\newcommand{\horrule}[1]{\rule{\linewidth}{#1}} % Create horizontal rule command with 1 argument of height
\newtheoremstyle{break}
  {\topsep}{\topsep}%
  {\itshape}{}%
  {\bfseries}{.}%
  {\newline}{}%
\theoremstyle{break}
\newtheorem{lem}{Lemma}
\newtheorem{defn}{Definition}
\newtheorem{thm}{Theorem}
\newtheorem{cor}{Corollary}
\newtheorem{prop}{Proposition}
\newtheorem{ex}{Example}
\renewcommand\qedsymbol{//}
\newtheorem{lma}{Lemma}
\setenumerate[1]{label=\Roman*.}
\setenumerate[2]{label=\Alph*.}
\setenumerate[3]{label=\roman*.}
\setenumerate[4]{label=\alph*.}

\title{	
\normalfont \normalsize 
\textsc{Colorado State University Mathematics} \\ [25pt] % Your university, school and/or department name(s)
%\horrule{0.5pt} \\[0.4cm] % Thin top horizontal rule
\huge Higher Ramification Groups \\ % The assignment title
%\horrule{2pt} \\[0.5cm] % Thick bottom horizontal rule
}

\author{Dean Bisogno} % Your name

\date{\normalsize\today} % Today's date or a custom date

\begin{document}

\maketitle % Print the title
\section{Abstract}
Studying higher ramification groups immediately depends on some key ideas from valuation theory. With that in mind, this paper will outline the essential results from valuation theory before proceeding to higher ramification groups. Higher ramification groups arise when studying extensions of fraction fields of Dedekind rings, and the ramification of primes in subrings of the base field. In particular ramification groups can be useful when studying wildly ramified extensions. The goal in a wildly ramified extension is to find the order of higher ramification groups to correct the degree of the different. Another application of higher ramification groups is to study the subgroup structure of a Galois group. The higher ramification group filtration contains information about Sylow p-subgroup structure. The aim of this paper is to understand at a general level what higher ramification groups are, and investigate a couple examples of their applications.
\section{Preliminaries}
\subsection{Valuations}
Valuations need not be discrete. Many results discussed in this paper either require discrete valuations or are most easily understood with respect to a discrete valuation. Thus discrete valuations will be the focus of this section.
\begin{defn}[Discrete Valuation Ring (D.V.R.)]
A discrete valuation ring is a principal ideal domain $O$ with a unique maximal ideal $p \not= 0$.
\end{defn}

\begin{ex} Consider the ring $\mathbb{Z}$ and prime $\mathfrak{p}\subset \mathbb{Z}$. The localization $\mathbb{Z}_{(\mathfrak{p})}$ is a local PID with unique maximal ideal $p\mathbb{Z}_{(\mathfrak{p})}$. Then $\mathbb{Z}_{(\mathfrak{p})}$ satisfies the definition of a discrete valuation ring.
\end{ex}
\begin{ex} Consider $k[[x]]$, the formal power series of the field $k$. It was shown in Math 566 (\cite{DnF}, 7.3 \#3) that all proper ideals of $k[[x]]$ are of the form $(x^n)$ (this is actually a corollary to what is shown in that problem). Further $(x)$ is maximal and contains all other proper ideals. It follows then that $(x)$ is the unique maximal ideal in $k[[x]]$ and all other proper ideals are principal. Thus $k[[x]]$ is a DVR.
\end{ex}
\begin{defn}[Uniformizing Parameter]
Let $\mathfrak{p}$ be the unique maximal ideal of D.V.R. $O$. Since $O$ is a PID, there exists a prime $\pi$ in $O$ such that $\mathfrak{p}=(\pi)$. Such a $\pi$ is called a uniformizing parameter.
\end{defn}
Theorem 7, section 16.2 in (\cite{DnF}) shows that $\pi$ is unique (up to multiplication by a unit) and generates the unique maximal ideal of $O$. Further in problem 26, in section 7.1 of \cite{DnF} (proven in math 566) it is shown that $O \setminus \mathfrak{p}$ is the set of units in $O$. Then for any $a\in O\setminus \{0\}$, $(a)$ is an ideal of $O$. Because $(a)$ is a nonzero ideal of $O$, $(\pi)$ is maximal, and $(a)$ is in a local ring $O$, there exists some $n\in \mathbb{Z}$ such that $(a) = (\pi)^n$. Notice if $a$ is a unit, then $n=0$ because $<a>=O$. 
\begin{defn}[Discrete Valuation]
The exponent $n$ used above is the valuation of $a$ denoted $v_\mathfrak{p}(a)$, it is a function $v_\mathfrak{p}:K^* \to \mathbb{Z}$ where $K$ is the field of fractions of $O$. The valuation $v_\mathfrak{p}$ can be extended to $K$ by setting $v_\mathfrak{p}(0) = \infty$. Further $v_\mathfrak{p}$ satisfies the following
\begin{enumerate}
\item $v_\mathfrak{p}(ab)=v_\mathfrak{p}(a) + v_\mathfrak{p}(b)$.
\item $v_\mathfrak{p}(a+b) \geq min\{v_\mathfrak{p}(a),v_\mathfrak{p}(b) \}$.
\end{enumerate}
\end{defn}
There are also valuations which are not discrete, and all valuations can be classified as either archimedean or nonarchimedean. A valuation $v$ is nonarchimedean if $v(n)$ is bounded for every $n\in \mathbb{N}$. Proposition 3.6 in \cite{Neukirch} is often incorporated in the definition of discrete valuations, as every discrete valuation is nonarchimedean.
\begin{prop}
 A valuaion $v(x)$ is nonarchimedean if and only if it satisfies $v(x+y) \leq max\{v(x),v(y)\}$.
\end{prop}
\begin{proof}
$(\Leftarrow)$ The reverse direction is straight forward. Just notice 
$$v(n)=v(1+\ldots+1) \leq 1\; \forall n\in\mathbb{N}.$$

$(\Rightarrow)$ Now, $v(n) \leq N$ for all $n\in\mathbb{N}$ for some $N \in \mathbb{N}$. Then for arbitrary $x,y\in K$, without loss of generality, consider $v(x) \geq v(y)$. Choose $l \geq 0$, then $v(x)^{l}v(y)^{n-l} \leq v(x)$. Now applying binomial formula yields
$$
	v(x+y)^{n} \leq \sum_{l=0}^{n} v\tbinom{n}{l} v(x)^l v(y)^{n-l} \leq N^{}(n+1)(v(x))^n
$$
taking the $n^{th}$ root of both sides yields
$$
v(x+y) \leq N^{\frac{1}{n}}(1+n)^{\frac{1}{n}} v(x) = N^{\frac{1}{n}}(1+n)^{\frac{1}{n}} max\{v(x), v(y) \}
$$

The result then follows if letting $n \to \infty$. The conclusion is then for any discrete valuation (which is nonarchimedean) $v(x+y) \leq max \{v(x),v(y) \}$.
\end{proof}

The following proposition is particularly useful because it provides a correspondence between prime ideals and valuations in Dedeking rings.
\begin{prop}[\cite{Neukirch}]
Let $O$ be a noetherian integral domain. $O$ is a Dedekind domain if and only if, for all prime ideals $\mathfrak{p} \not= 0$, the localizations $O_\mathfrak{p}$ are discrete valuation rings.
\end{prop}
When the previous proposition if applied to $\mathbb{Z}$ and the fraction field is consider, the following proposition is gained.
\begin{prop}[\cite{Neukirch}]
Every valuation of $\mathbb{Q}$ is equivalent to one of $\nu_p(x)$ (the nonarchimedean valuations) or $\nu_\infty(x)$ (the archimedean valuations).
\end{prop}
Finally, since ramification occurs via extensions or covers, it will be useful to know how valuations extend.
\begin{defn}[Prolongation of a valuation]
If $A \subset B$ are rings with $B$ integral over $A$, $\mathfrak{p}$ a prime in $A$, and $\mathfrak{q}$ prime in $B$ dividing $\mathfrak{p}$ with ramification index $e_\mathfrak{q}$ then $v_\mathfrak{q}(x)=e_\mathfrak{q}v_\mathfrak{p}(x)$ is a prolongation of $v_\mathfrak{p}$ with index $e_\mathfrak{q}$ (\cite{Serre}).
\end{defn}

\subsection{Field Completion}
With a valuation $\nu$ on a field $K$ then for any real number $a\in(0,1)$ an absolute value on $K$ can be defined by
\[ ||x|| =  \begin{cases} 
      a^{\nu(x)} & x \not= 0 \\
      0 & x = 0 \\
   \end{cases}
\]
which satisfies the usual conditions of a metric as outlined by \cite{Serre} section II.1. A topology is then induced on $K$ via the absolute value metric, and the completion of $K$ with respect to the valuation $\nu$ is denoted by $\widehat{K}$. Note also that the metrics induced by different choices of $a$ are topologically equivalent, so the completion is dependent only on $\nu$. Further, $\nu$ extends in the completion of $K$ to a discrete valuation (which will continue to be called $\nu$) on $\widehat{K}$ (\cite{Serre}).

The Galois group of $\widehat{L}/\widehat{K}$ relates to the decomposition group of $L/K$ in the following way.
\begin{prop}
For $L/K$ is Galois with group $\Gamma$, $\nu_\mathfrak{p}(x)$ a valuation on $K$ and $\omega_\mathfrak{q}(x)$ a prolongation of $\nu_\mathfrak{p}$, let $\widehat{L}$,$\widehat{K}$ be the completions of $L$ and $K$ with respect to $\omega_\mathfrak{q}$ and $\nu_\mathfrak{p}$ respectively. If $D(L/K)$ is the decomposition group of $\mathfrak{q}$ over $\mathfrak{p}$, then $\widehat{L}/\widehat{K}$ is Galois with Galois group $D(L/K)$ \cite{Serre}.
\end{prop}
\begin{proof}
This is corollary 4 to theorem 1 in \cite{Serre}, II.3 which shows that $[\widehat{L}:\widehat{K}]=ef$. As proven in class $D(L/K)=ef$.
\end{proof}
\section{Higher Ramification Groups}
Consider now $K$, the fraction field of a Dedekind ring $O_K$, with valuation $\nu$ and uniformizing parameter $\pi_K$. For Galois extension $L/K$ with Galois group $\Gamma$; prolongation of $\nu$, $\omega$, on $L$; uniformizing parameter $\pi_L$ of $\omega$; and discrete valuation ring $O_L$. The definitions of the higher ramification groups generalizes the definition of the inertia group.
\begin{defn}[Higher Ramification Groups, \cite{Serre}]
For every real number $i > -1$ define the $i^{th}$ ramification group of $L/K$ by
$$
G_i = G_i(L/K) = \{\sigma \in \Gamma | \omega(\sigma(a)-a)\geq i+1 \;\forall a \in O_L\}.
$$
\end{defn}
The following is a proposition which helps characterize the higher ramification groups with respect to the presentation of valuations above.
\begin{prop}
For any $a \in O_L$ and $\sigma \in \Gamma$, $\omega(\sigma(a) - a) \geq i+1$ if and only if  $\sigma(a) \equiv a \; mod \; (\pi_L)^{i+1}$.
\end{prop}
\begin{proof}
\mbox{}\\*
($\Rightarrow$)
$$
\nu_L(\sigma(a)-a) \geq i+1 \implies \sigma(a)-a = \pi_L^{t}\frac{x}{y}
$$
where $t \geq i+1$ and $x,y,\pi_L\in O_L$ all relatively prime. This follows from the definition of the valuation in the first section. The equation above then implies $\sigma(a)-a \equiv 0 \; mod \; (\pi_L)^{i+1}$.

($\Leftarrow$) $$\sigma(a)-a \equiv 0 \; mod \; \pi_L^{i+1} \implies \sigma(a)-a = \pi_L^{t}\frac{x}{y}$$
with the same restrictions as above. It follows then that $\nu_L(\sigma(a)-a) \geq i+1$.
\end{proof}
The condition that $\nu(\sigma(a)-a) \geq i+1$ clearly induces the following filtration on $\Gamma$
$$
\{G_i(L/K)\}_{i\geq -1} = G_{-1} \supseteq G_0 \supseteq G_1 \supseteq \ldots.
$$
This filtration can be investigated further due to the fact that $\forall i$, $G_i \triangleleft G$.
\begin{proof} Consider $\tau \in \Gamma$. By membership of $G$, $\omega(\tau(a) - a) \geq 0\;\forall a\in O_L$, i.e. $\tau((\pi_L)^{i+1}) = (\pi_L)^{i+1}$ for all $i$. Thus there is an induced automorphism $\bar{\tau}$ of $O_L/(\pi_L^{i+1})$. Now consider $\phi_{i+1}:G \to aut(O_L/(\pi_L^{i+1})$ such that $\phi_{i+1}(\tau)=\bar{\tau}$. This is a group homomorphism with $ker(\phi_{i+1})=G_i$, so $G_i \triangleleft \Gamma$ for all $i$.
\end{proof} 
The first two ramification groups, $G_{-1}$ and $G_0$, are already familiar. For $i=-1$ from the definition it follows $G_{-1} = \{\sigma \in \Gamma |\omega(\sigma(a)-a) \geq 0\;\forall a\in O_L\}=\Gamma$ because $O_L$ is the DVR of $L$ with respect to $\omega$. For $i+0$, $G_{0}$ equals the inertia group $I(L/K)$ by noticing that $\phi_1$ is the homomorphism used to originally define the inertia group.

Occasionally it is useful to denote the first index of the ramification group in which an element $\sigma\in\Gamma$ ceases to appear. 
\begin{defn}
Let $x\in O_L$ such that $O_L=O_K[x]$. Such an element is assured to exist by \cite{Neukirch}, II.10.4. Then $\sigma\in\Gamma$ define
$$
i_{L/K}(\sigma) = \omega(\sigma(x)-x).
$$
\end{defn}
Noting the argument above which showed $G_i \triangleleft \Gamma$, it follows $\forall \sigma,\tau \in \Gamma$, $i_{L/K}(\tau^{-1}\sigma \tau) = i_{L/K}(\sigma)$. Another simple conclusion which follows from the definition of $i_{L/K}$ is $i_{L/K}(\sigma)\geq i+1 \iff \sigma\in G_i$.

Consider now $H\subset\Gamma$ a subgroup, and let $K'$ be the fixed field of $H$ so that $H=Gal(L/K')$. The following relations follow from the above propositions.
\begin{prop}
For every $\sigma\in H$, $i_{L/K'}(\sigma)=i_{L/K}(\sigma)$, and $H_i=G_i \cap H$ (\cite{Serre}).
\end{prop}
\begin{proof}
Let $\phi_{i+1}:\Gamma \to Aut(O_L/(\pi_L)^{i+1})$ and $\psi_{i+1}:H \to Aut(O_L/(\pi_L)^{i+1})$ such that $\phi_{i+1}(\sigma)=\bar{\sigma}$ and $\psi_{i+1}$ defined similarly. As shown above, $G_i = ker(\phi_{i+1})$ and $H_i = ker(\psi_{i+1})$. Thus the conclusions $H_i = H\cup G_i$ and $i_{L/K'}(\sigma)=i_{L/K}(\sigma)$ follow.
\end{proof}


\section{The Upper Numbering and Ramification Jump}
There is a renumbering of the ramification groups which lends itself to other computations. The ordering used above is called the lower numbering while the renumbering introduced below is called the upper numbering. The upper numbering is useful when working with quotient group computations, while the lower numbering works best for subgroup computations (\cite{Neukirch}).
\begin{defn}[Upper Numbering of the Higher Ramification groups]
Consider the function
$$
t = \varphi(s) = \int_{0}^{s} \frac{dx}{[G_0 : G_x]}
$$
called the Herbrand function which has inverse map $\psi$.
Then for any real $s \geq -1$ let $G_s = G_{\lceil s \rceil}$ and renumber the ramification groups by $G^t(L/K) = G_s(L/K)$ where $s=\psi(t)$. 
\end{defn}
As this new numbering allows for noninteger index, the index at which the ramification group changes is given a name.
\begin{defn}[Ramification Jump]
If $G^{t}(L/K) \not=G^{t+\epsilon}(L/K)$ for any $\epsilon \geq 0$ then $t$ is called a ramification jump.
\end{defn}
These two concepts are important for the final theorem of this paper. Which assures that in certain circumstances, ramification groups only change at integer indexes.
\begin{thm}[Hasse-Arf]
For a finite abelian extension $L/K$, the jumps of the filtration $\{G^{i}(L/K)\}_{i \geq -1}$ are rational integers.
\end{thm}
Serre gives a useful interpretation of the Hasse-Arf theorem, namely that if $G_i \not= G_{i+1}$ then $\varphi(i)$ is an integer. This interpretation is helpful in the following example (\cite{Serre}).

\begin{ex}
Suppose $G$ is a cyclic group of order $p^n$ where $p$ is the characteristic of $\bar{K}$. Let $G(i)$ be the subgroup of $G$ with order $p^{n-i}$. Then there exist integers $i_0, \ldots,i_{n-1}$ such that all ramification groups are identified as follows:
$$G_0 = G_{i_0} = G = G^0 = G^{i_0}$$
$$G_{{i_0}+1} = \ldots = G_{{i_0}+pi_1} = G(1) = G^{i_0+1} = \ldots = G^{i_0 + i_1}$$
$$G_{{i_0}+pi_1+1} = \ldots = G_{{i_0}+pi_1+p^2 i_2} = G(2) = G^{i_0+i_1+1} = \ldots = G^{i_0 + i_1+i_2}$$
$$\vdots$$
$$G_{i_0+pi_1+p^2 i_2 + \cdots + p^{n-1}i_{n-1} +1}={1}=G^{i_0+\cdots+i_{n-1}+1}.$$
\end{ex}

\begin{proof}
 Because $G=Gal(\bar{L}/\bar{K})$ is cyclic with order $p^n$, it follows that $\bar{L}/\bar{K}$ is an extension of finite fields, i.e. $L\cong \bar{\mathbb{F}}_{p^n}$ and $K\cong \bar{\mathbb{F}}_p$. Further recall that $G$ is generated by fr\"{o}benius $\sigma$ (shown in Math 567). By Euler's totient there are $(p-1)p^n$ elements in $G$ with order relatively prime to $p^n$, consequently since $p^n-(p-1)p^{n-1}=p^{n-1}$, there are $n-1$ distinct proper subgroups of $G$. This places $n$ as an upper bound on the number of distinct ramification groups.

Consider \cite{stitchenoth}, III.8.6 which states that if $char{\bar{K}} =p>0$, then $G_{i+1} \triangleleft G_i$ for all $i$. Then since each $G_i$ is a subgroup of a cyclic group, for each ramification group there exists a $\sigma^l\in G$ which generate $G_i$. As argued above there are only $n-1$ such $\sigma^{l}$ generating distinct proper subgroups of G. This places $n$ as the lower bound on the number of distinct ramification groups. Thus there are $n-1$ ramification jumps and $G_i = (\sigma^l)$ where $0 \leq l < n$.

Finally $(\sigma^l)=Gal(\bar{\mathbb{F}}_{p^n}/\bar{\mathbb{F}}_{p^l})$ which is isomorphic to the cyclic group of order $p^{n-l}$.
\end{proof}

\section{Applications}
\subsection{Cyclotomic Extensions of $\mathbb{Q}_p$}
In this example consider the p-adic completion of $\mathbb{Q}$ with respect to a valuation (and associated metric) $\nu_p$. Let $n=p^m$, and $\zeta$ be a primitive $n^{th}$ root of unity and the extension $\mathbb{Q}_p(\zeta)/\mathbb{Q}_p$ with Galois group $G$. Recall that $[\mathbb{Q}_p(\zeta):\mathbb{Q}_p]=\phi(n)=(p-1)p^{m-1}$ and $G\cong (\mathbb{Z}/n)^*$. 

Now, if $0 \leq v \leq m$ then let $G^v$ be the subgroup of $G$ isomorphic to the subgroup $H \subset (\mathbb{Z}/n)^*$ such that $a \equiv 1\;mod\;p^v$ for every $a\in H$. Then $Gal(\mathbb{Q}_p(\zeta_{p^m})/\mathbb{Q}_p(\zeta_{p^v})) \cong G^v$ because $Gal(\mathbb{Q}_p(\zeta_{p^v})/\mathbb{Q}_p) \cong (\mathbb{Z}/p^v)^*$. Using this fact it is straight forward to find all ramification groups $G_i$ of $G$ (\cite{Serre}).
\begin{prop}
The ramification groups $G_i$ of $G$ are
\[ G_u = \begin{cases} 
      G    & u=0 \\
      G^1  & 1 \leq u \leq p -1 \\
      G^2  & p \leq u \leq p^2 -1\\
      \vdots \\
      \{1\} & p^{m-1} \leq u
   \end{cases}
\]
\end{prop}

\begin{proof}
Consider an element $x\not=1$ in $G(n)\cong (\mathbb{Z}/n)^*$ and $\sigma_x$ the corresponding element in $G$. Let $i\in\mathbb{Z}$ be maximal such that $x \equiv 1 \; mod\;p^i$. By definition of $G(n)^i$ and maximality of $i$, $x\in G(n)^i$ but $x\not\in G(n)^{i+1}$. Now consider
$$
i_{\mathbb{Q}_p(\zeta)/\mathbb{Q}_p}(\sigma_x) = \omega(\sigma_x(\zeta)-\zeta) = \omega(\zeta^q-\zeta)=\omega(\zeta^{q-1}-1).
$$
Because $\zeta^{q-1}$ is a primitive $p^{m-i}$ root of unity, $\zeta^{q-1}-1$ is a uniformizer of the valuation on $\mathbb{Q}_p(\zeta_{p^{m-i}})$ (this is \cite{Serre}, prop. IV.17.iii). Then it follows
$$
i_{\mathbb{Q}_p(\zeta)/\mathbb{Q}_p}(\sigma_x) = [\mathbb{Q}_p(\zeta_{p^{m}}):\mathbb{Q}_p(\zeta_{p^{m-i}})]=\#(G(n)^{m-i})=p^i.
$$
If $p^{k-1} \leq u \leq p^k -1$, then $\sigma_x\in G_u$ if and only if $i\geq k$. Consequently $G_u = G(n)^i$.
\end{proof}

\subsection{Artin-Schreier Extensions in Positive Characteristic}
The Riemann-Hurwitz formula can be generalized in terms of Weil differentials which are beyond the scope of this paper. Such a generalization yields the following
$$
2g'-2=\frac{[F':F]}{[K':K]}(2g-2) + deg(Diff(F'/F)).
$$
Where $F/K$ an algebraic function field of genus $g$ and $F'/F$ a finite separable extension. Let $K'$ denote the constant field of $F'$ and $g'$ the genus of $F'/K'$ (\cite{stitchenoth}).
Riemann-Hurwitz such that is a finite, or algebraically closed field $\mathbb{F}$ has characteristic $p$ and $p|e$ where $e$ is the ramification index of a point under a degree $d$ covering map $\phi:X \to Y$ of curves over ($\mathbb{F}$ \cite{RnR}). 
$$
2g_Y -2 = d(2g_X -2) + \sum_{x\in X} \big( \sum^{\infty}_{i=0}|G_i(x)| -1 \big).
$$

Consider a smooth compact algebraic curve $X_{p,t}: x^p - x - f(y)$, $f(y)$ a degree $t$ polynomial where $p\not|t$. Further consider the covering map $\psi: X_{p,t} \to \mathbb{A}^1_\mathbb{F}$ such that $\psi(x,y)=y$. 
The case considered here is when $X_{p,t}: x^p-x = y^t$ where $t=ap-1$.
Strategy (use change of variables $x=x\bar{y}^a$,$\bar{y}=1/y$ so $p_\infty = (0,0)$, and $t=ap-1$):
Since $\psi$ is clearly unramified over $\mathbb{A}^1_\mathbb{F}$, consider now $\psi: X_{p,t} \to \mathbb{P}^1_\mathbb{F}$. Then $\psi$ is only ramified over $p_\infty$. 


In order to compute the ramification groups it necessary to understand the Galois group of $\psi$. To do so consider $Auth(X)$. Let $\tau(x)=x+1$ then if $x^p - x - f(y) = 0$ it follows
$$
\tau(x)^p - \tau(x) - f(y) = (x+1)^p - (x+1) - f(y) = x^p + 1 - x - 1 - f(y) = x^p-x-f(y) = 0
$$
because $char(\mathbb{F})=p$. Consequently $\tau\in Aut(X_{p,t})$ and in fact generates the automorphism group which is isomorphic to $(\mathbb{Z}/p)^*$ which agrees with the conclusion of \cite{stitchenoth}, III.7.10. Let $G=Aut(\psi)$.

Applying the change of variables $\bar{y}=\frac{1}{y}$, $\bar{x}=x\bar{y}^a$ will make computations simpler because then $p_\infty = (0,0)$. The change of variables yields:
$$x^p-x-y^t = 0 \implies x^p-x-y^{ap-1}=0 \implies x^p-x-y^{ap}y^{-1}=0 $$
$$\implies x^p y^{-ap} -xy{-ap}-y^{-1}=0 \implies (xy^{-a})^p - xy^{-a}y^{-a(p-1)}-y^{-1}=0 $$
$$\implies \bar{x}^p-\bar{x}\bar{y}^{a(p-1)}-\bar{y} = 0.$$
Thus $X^{\bullet}_{p,t}: \bar{x}^p-\bar{x}\bar{y}^{a(p-1)}-\bar{y}$ which is ramified over $p_\infty = (0,0)$.

Notice now that $\nu_\infty(\bar{x}) = 1$ because $\bar{x}$ vanishes with order 1 at $p_\infty$ so $\pi_\infty = \bar{x}$ is the uniformizing parameter over $P_\infty$. Further applying $\tau$, the generator of $G$, to $\bar{x}$ results in the following:
$$\tau(\bar{x}) = (x+1)\bar{y}^a = \bar{x} + \bar{y}^a$$
$$\implies \tau(\pi)-\pi =  \tau(\bar{x}) - \bar{x} = \bar{y}^a.$$
Consequently it is possible to compute $i_G(\tau)$ (recall this is the index of the first ramification group in which $\tau$ no longer appears):
$$\nu_\infty(\tau(\pi)-\pi) = \nu_\infty(\bar{y}^a)=a\nu_\infty(\bar{y})$$
$$a\nu_\infty(\bar{y})=a\nu_\infty(\bar{x}^p - \bar{x}\bar{y}{a(p-1)})=a\cdot min\{\nu_\infty(\bar{x}^p),\nu_\infty(\bar{x}\bar{y}^{a(p-1)})\}$$
$$\nu_\infty(\tau(\pi)-\pi) = \nu_\infty(\bar{y}^a) = a\cdot\nu_\infty(\bar{x}^p)=ap.$$

What has been calculated is the ramification jump $t_{p_\infty} = ap-1$ meaning $G = G_0 \ldots G_{ap-1} \triangleright G_{ap} = G{ap+1}\ldots$. Because $(\mathbb{Z}/p)^*$ is simple and $G_{ap} \triangleleft G$ this implies that $G_{ap} = \{1\}$ and all previous ramification groups are isomorphic to $(\mathbb{Z}/p)^*$. Consequently it is now possible to calculate the degree of the different
$$
\sum_{b\in\mathbb{A}^1_\mathbb{F}} \sum_{i=0}^\infty |G_i| - 1 = \sum_{b\in\mathbb{A}^1_\mathbb{F}} (p-1)(ap).
$$
Where $|(\mathbb{Z}/p)^*|=(p-1)$ and $ap$ is the number of nontrivial ramification groups. Because $p_\infty$ is the only ramified point, the above reduces to $(p-1)(ap)$ as all other points have trivial ramification groups. Thus the genus of $X^\bullet_{p,t}$ can be accurately calculated (using $ap=t+1$:
$$
2g_{X_{p,t}} -2= p(2g_{\mathbb{P}^1_{\mathbb{F}}} -2) - (p-1)(ap) \implies  g_{X_{p,t}} = \frac{-2p-(p-1)(t+1)+2}{2}=\frac{(p-1)(t+1)}{2}.
$$
This result can be confirmed using the standard Riemann-Hurwitz formula for the covering map $\phi:X_{p,t} \to \mathbb{P}_\mathbb{F}^1$ where $\phi(x,y)=x$ which is ramified when $x\in \mathbb{Z}/p$ and $y=0$ with ramification index $t$ not divisible by $p$.
$$
g_{X_{p,t}} = \frac{p(-2) + p(t-1) +2}{2} = \frac{(p-1)(t+1)}{2}.
$$
Which agrees with the previous computation.
\clearpage
%\nocite{*}
\printbibliography


\end{document}