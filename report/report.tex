%%%%%%%%%%%%%%%%%%%%%%%%%%%%%%%%%%%%%%%%%
% Short Sectioned Assignment
% LaTeX Template
% Version 1.0 (5/5/12)
%
% This template has been downloaded from:
% http://www.LaTeXTemplates.com
%
% Original author:
% Frits Wenneker (http://www.howtotex.com)
%
% License:
% CC BY-NC-SA 3.0 (http://creativecommons.org/licenses/by-nc-sa/3.0/)
%
%%%%%%%%%%%%%%%%%%%%%%%%%%%%%%%%%%%%%%%%%

%----------------------------------------------------------------------------------------
%	PACKAGES AND OTHER DOCUMENT CONFIGURATIONS
%----------------------------------------------------------------------------------------

\documentclass[paper=a4, fontsize=11pt]{scrartcl} % A4 paper and 11pt font size

\usepackage[T1]{fontenc} % Use 8-bit encoding that has 256 glyphs
\usepackage{fourier} % Use the Adobe Utopia font for the document - comment this line to return to the LaTeX default
\usepackage{amsmath,amsfonts,amsthm} % Math packages
\usepackage[american]{babel}
\usepackage{csquotes}
\usepackage[style=apa,sortcites=true,sorting=nyt,backend=biber]{biblatex} 
\DeclareLanguageMapping{american}{american-apa}%bibliography package

\usepackage{sectsty} % Allows customizing section commands
\allsectionsfont{\centering \normalfont\scshape} % Make all sections centered, the default font and small caps

\usepackage{fancyhdr} % Custom headers and footers
\pagestyle{fancyplain} % Makes all pages in the document conform to the custom headers and footers
\fancyhead{} % No page header - if you want one, create it in the same way as the footers below
\fancyfoot[L]{} % Empty left footer
\fancyfoot[C]{} % Empty center footer
\fancyfoot[R]{\thepage} % Page numbering for right footer
\renewcommand{\headrulewidth}{0pt} % Remove header underlines
\renewcommand{\footrulewidth}{0pt} % Remove footer underlines
\setlength{\headheight}{13.6pt} % Customize the height of the header

\usepackage{enumitem}
\numberwithin{equation}{section} % Number equations within sections (i.e. 1.1, 1.2, 2.1, 2.2 instead of 1, 2, 3, 4)
\numberwithin{figure}{section} % Number figures within sections (i.e. 1.1, 1.2, 2.1, 2.2 instead of 1, 2, 3, 4)
\numberwithin{table}{section} % Number tables within sections (i.e. 1.1, 1.2, 2.1, 2.2 instead of 1, 2, 3, 4)

\setlength\parindent{0pt} % Removes all indentation from paragraphs - comment this line for an assignment with lots of text
\bibliography{biblio}
%----------------------------------------------------------------------------------------
%	TITLE SECTION
%----------------------------------------------------------------------------------------

\newcommand{\horrule}[1]{\rule{\linewidth}{#1}} % Create horizontal rule command with 1 argument of height
\newtheoremstyle{break}
  {\topsep}{\topsep}%
  {\itshape}{}%
  {\bfseries}{.}%
  {\newline}{}%
\theoremstyle{break}
\newtheorem{lem}{Lemma}
\newtheorem{defn}{Definition}
\newtheorem{thm}{Theorem}
\newtheorem{cor}{Corollary}
\newtheorem{prop}{Proposition}
\newtheorem{ex}{Example}
\renewcommand\qedsymbol{//}
\newtheorem{lma}{Lemma}
\setenumerate[1]{label=\Roman*.}
\setenumerate[2]{label=\Alph*.}
\setenumerate[3]{label=\roman*.}
\setenumerate[4]{label=\alph*.}

\title{	
\normalfont \normalsize 
\textsc{Colorado State University Mathematics} \\ [25pt] % Your university, school and/or department name(s)
%\horrule{0.5pt} \\[0.4cm] % Thin top horizontal rule
\huge Higher Ramification Groups \\ % The assignment title
%\horrule{2pt} \\[0.5cm] % Thick bottom horizontal rule
}

\author{Dean Bisogno} % Your name

\date{\normalsize\today} % Today's date or a custom date

\begin{document}

\maketitle % Print the title
\section{Abstract}
\section{Definitions and Notation}
\subsection{Discrete Valuations}
We borrow our definitions in this section from \cite{DnF}. 
\begin{defn}[Discrete Valuations]
A discrete valuation on a field $K$ is a surjective function $\nu_K: K^* \to \mathbb{Z}$ such that
\begin{enumerate}
\item $\nu_K(xy) = \nu(x) + \nu(y)$ for all $x,y\in K^X$.
\item $\nu_K(x+y) \geq min\{\nu_K(x),\nu_K(y)\}$ for all $x,y\in K^X$ such that $x+y \not= 0$.
\end{enumerate}
Further, we call the subring $\{x \in K|\nu_K(x) \geq 0\} \cup \{0\}$ the valuation ring of $K$.
\end{defn}
\begin{defn}[Discrete Valuation Ring (D.V.R.)]
If $R$ is an integral domain and $R$ is the valuation ring of the field of fractions on $R$, then we call $R$ a discrete valuation ring. The ring $R$ also has a local ring with unique maximal ideal $M=\{r \in R | \nu_K(x) > 0\}$.
\end{defn}
Note that in Math 566 we proved (exercise 26 section 7.1) that for any unit $u \in K^*$ that $\nu_K(u)=0$. Thus it follows that $R \setminus M$ is the set of invertible elements of $R$.
\begin{defn}[Uniformizing Parameter]
If $R$ is a D.V.R. of $K$ via $\nu_K$, then an element $t\in R$ such that $\nu_K(t) = 1$ is a uniformizing parameter for R. Theorem 7, section 16.2 in \cite{DnF} shows that $t$ is unique (up to multiplication by a unit) and generates the unique maximal ideal of $R$.
\end{defn}

\subsection{Field Completion}
With a valuation $\nu_K$ on a field $K$ then for any real number $a\in(0,1)$ we can induce an absolute value on $K$ by

\[ ||x|| =  \begin{cases} 
      a^{\nu_K(x)} & x \not= 0 \\
      0 & x = 0 \\
   \end{cases}
\]
which satisfies the usual conditions of a metric as outlined by \cite{Serre}. A topology is then induced on $K$ via the absolute value metric, and we can denote the completion of $K$ with respect to the valuation $\nu_K$ by $K_\nu$. Note also that the metrics induced by different choices of $a$ are topologically equivalent, so the completion is dependent only on $\nu_K$. Further, $\nu_K$ extends in the completion of $K$ to a discrete valuation (which we will continue to call $\nu_K$) on $K_\nu$ (\cite{Serre}).

\subsection{Higher Ramification Groups}
Consider now a finite Galois extension $L|K$ with Galois group $G$, and associated discrete valuations $\nu_K$ and $\nu_L$ and valuation rings $O_L$, and $O_K$.
\begin{defn}[Higher Ramification Groups]
For every real number $i > -1$ we define the $i^{th}$ ramification group of $L|K$ by
$$
G_i = G_i(L|K) = \{\sigma \in G | \nu_L(\sigma(a)-a)\geq i+1 \;\forall a \in O_L\}.
$$
\end{defn}

Clearly this induces the following filtration on G:
$$
\{G_i(L|K)\}_{i\geq -1} = G_-1 \supseteq G_0 \supseteq G_1 \supseteq \ldots
$$

\begin{prop}
For any $a \in O_L$ and $\sigma \in G$, $\nu_L(\sigma(a) - a) \geq i+1$ if and only if  $\sigma(a) \cong a \; mod \; P_L^{i+1}$.
\end{prop}


\begin{prop}
The first two ramification groups are $G_{0} = G$ and $G_{1}=I$.
\end{prop}

\section{The Upper Numbering and Ramification Jump}

\begin{defn}[Upper Numbering the Higher Ramification groups]
Consider the function
$$
t = \varphi(s) = \int_{0}^{s} \frac{dx}{[G_0 : G_x]}
$$
called the Herbrand function which has inverse map $\psi$.
Then for any real $s \geq -1$ let $G_s = G_{\lceil s \rceil}$ and renumber the ramification groups by $G^t(L|K) = G_s(L|K)$ where $s=\psi(t)$. 
\end{defn}
\begin{defn}[Ramification Jump]
If $G^{t}(L|K) \not=G^{t+\epsilon}(L|K)$ for any $\epsilon \geq 0$ then we call $t$ a ramification jump.
\end{defn}
\begin{thm}[Hasse-Arf]
For a finite abelian extension $L|K$, the jumps of the filtration $\{G^{i}(L|K)\}_{i \geq -1}$ are rational integers.
\end{thm}
\begin{ex}
\end{ex}

\subsection{Application to Wild Riemann-Hurwitz}
We can rephrase Riemann-Hurwitz such that if field $\mathbb{F}$ has characteristic $p$ and $p|e$ where $e$ is the ramification index of a point under a degree $d$ covering map $\phi:X \to Y$ of curves over $F$
Then
$$
2g_Y -2 = d(2g_X -2) + \sum_{x\in X} \sum^{\infty}_{i=0}|G_i(x)| -1
$$
\nocite{*}
\printbibliography


\end{document}