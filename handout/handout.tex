%%%%%%%%%%%%%%%%%%%%%%%%%%%%%%%%%%%%%%%%%
% Short Sectioned Assignment
% LaTeX Template
% Version 1.0 (5/5/12)
%
% This template has been downloaded from:
% http://www.LaTeXTemplates.com
%
% Original author:
% Frits Wenneker (http://www.howtotex.com)
%
% License:
% CC BY-NC-SA 3.0 (http://creativecommons.org/licenses/by-nc-sa/3.0/)
%
%%%%%%%%%%%%%%%%%%%%%%%%%%%%%%%%%%%%%%%%%

%----------------------------------------------------------------------------------------
%	PACKAGES AND OTHER DOCUMENT CONFIGURATIONS
%----------------------------------------------------------------------------------------

\documentclass[paper=a4, fontsize=11pt]{scrartcl} % A4 paper and 11pt font size

\usepackage[T1]{fontenc} % Use 8-bit encoding that has 256 glyphs
\usepackage{fourier} % Use the Adobe Utopia font for the document - comment this line to return to the LaTeX default
\usepackage{amsmath,amsfonts,amsthm} % Math packages
\usepackage[american]{babel}
\usepackage{csquotes}
\usepackage[style=apa,sortcites=true,sorting=nyt,backend=biber]{biblatex} 
\DeclareLanguageMapping{american}{american-apa}%bibliography package

\usepackage{sectsty} % Allows customizing section commands
\allsectionsfont{\centering \normalfont\scshape} % Make all sections centered, the default font and small caps

\usepackage{fancyhdr} % Custom headers and footers
\pagestyle{fancyplain} % Makes all pages in the document conform to the custom headers and footers
\fancyhead{} % No page header - if you want one, create it in the same way as the footers below
\fancyfoot[L]{} % Empty left footer
\fancyfoot[C]{} % Empty center footer
\fancyfoot[R]{\thepage} % Page numbering for right footer
\renewcommand{\headrulewidth}{0pt} % Remove header underlines
\renewcommand{\footrulewidth}{0pt} % Remove footer underlines
\setlength{\headheight}{13.6pt} % Customize the height of the header

\usepackage{enumitem}
\numberwithin{equation}{section} % Number equations within sections (i.e. 1.1, 1.2, 2.1, 2.2 instead of 1, 2, 3, 4)
\numberwithin{figure}{section} % Number figures within sections (i.e. 1.1, 1.2, 2.1, 2.2 instead of 1, 2, 3, 4)
\numberwithin{table}{section} % Number tables within sections (i.e. 1.1, 1.2, 2.1, 2.2 instead of 1, 2, 3, 4)

\setlength\parindent{0pt} % Removes all indentation from paragraphs - comment this line for an assignment with lots of text
\bibliography{biblio}
%----------------------------------------------------------------------------------------
%	TITLE SECTION
%----------------------------------------------------------------------------------------

\newcommand{\horrule}[1]{\rule{\linewidth}{#1}} % Create horizontal rule command with 1 argument of height
\newtheoremstyle{break}
  {\topsep}{\topsep}%
  {\itshape}{}%
  {\bfseries}{.}%
  {\newline}{}%
\theoremstyle{break}
\newtheorem{lem}{Lemma}
\newtheorem{defn}{Definition}
\newtheorem{thm}{Theorem}
\newtheorem{cor}{Corollary}
\newtheorem{prop}{Proposition}
\newtheorem{ex}{Example}
\renewcommand\qedsymbol{//}
\newtheorem{lma}{Lemma}
\setenumerate[1]{label=\Roman*.}
\setenumerate[2]{label=\Alph*.}
\setenumerate[3]{label=\roman*.}
\setenumerate[4]{label=\alph*.}

\title{	
\normalfont \normalsize 
\textsc{Colorado State University Mathematics} \\ [25pt] % Your university, school and/or department name(s)
%\horrule{0.5pt} \\[0.4cm] % Thin top horizontal rule
\huge Background on Valuations \\ % The assignment title
%\horrule{2pt} \\[0.5cm] % Thick bottom horizontal rule
}

\author{Dean Bisogno} % Your name

\date{\normalsize\today} % Today's date or a custom date

\begin{document}
\pagenumbering{gobble}
\maketitle
\begin{defn}[Discrete Valuation Ring (D.V.R.)]
A discrete valuation ring is a principal ideal domain $O$ with a unique maximal ideal $p \not= 0$.
\end{defn}

\begin{defn}[Uniformizing Parameter]
Let $\mathfrak{p}$ be the unique maximal ideal of D.V.R. $O$. Since $O$ is a PID, there exists a prime $\pi$ in $O$ such that $\mathfrak{p}=(\pi)$. Such a $\pi$ is called a uniformizing parameter.
\end{defn}
\begin{defn}[Discrete Valuation]
A discrete valuation is a function $v_\mathfrak{p}:K^* \to \mathbb{Z}$ where $K$ is the field of fractions of $O$. The valuation $v_\mathfrak{p}$ can be extended to $K$ by setting $v_\mathfrak{p}(0) = \infty$. Further $v_\mathfrak{p}$ satisfies the following
\begin{enumerate}
\item $v_\mathfrak{p}(ab)=v_\mathfrak{p}(a) + v_\mathfrak{p}(b)$.
\item $v_\mathfrak{p}(a+b) \geq min\{v_\mathfrak{p}(a),v_\mathfrak{p}(b) \}$.
\end{enumerate}
\end{defn}
\begin{prop}[Nonarchimedean]
 A valuaion $v(x)$ is nonarchimedean if and only if it satisfies $v(x+y) \leq max\{v(x),v(y)\}$. All discrete valuations are nonarchimedean.
\end{prop}
\begin{prop}
Let $O$ be a noetherian integral domain. $O$ is a Dedekind domain if and only if, for all prime ideals $\mathfrak{p} \not= 0$, the localizations $O_{(\mathfrak{p})}$ are discrete valuation rings.
\end{prop}
\begin{defn}[Prolongation of a valuation]
If $A \subset B$ are rings with $B$ integral over $A$, $\mathfrak{p}$ a prime in $A$, and $\mathfrak{q}$ prime in $B$ dividing $\mathfrak{p}$ with ramification index $e_\mathfrak{q}$ then $v_\mathfrak{q}(x)=e_\mathfrak{q}v_\mathfrak{p}(x)$ is a prolongation of $v_\mathfrak{p}$ with index $e_\mathfrak{q}$.
\end{defn}


\end{document}